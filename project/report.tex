\documentclass[sigconf]{acmart}

\input{format/i523}

\definecolor{light-gray}{gray}{0.95}
\usepackage{caption}
\usepackage{listings} %code extracts
\usepackage{xcolor} %custom colours
\usepackage{mdframed} %nice frames

\mdfsetup{skipabove=5pt,skipbelow=5pt}
\mdfdefinestyle{default}{%
backgroundcolor=light-gray, roundcorner=10pt,
leftmargin=1, rightmargin=1, innerleftmargin=10, 
innertopmargin=2,innerbottommargin=2, outerlinewidth=1,
linecolor=light-gray}
\lstset{breaklines=true, basicstyle=\tiny}

\begin{document}
\title{The Impact of Clinical Trial Results on Pharmaceutical Stock Performance}

\author{Tiffany Fabianac} 
 \affiliation{% 
   \institution{Indiana University} 
   \city{Bloomington}  
   \state{Indiana}  
   \postcode{47408} 
   \country{USA}
 } 
 \email{tifabi@iu.edu} 
 \renewcommand{\shortauthors}{T. Fabianac} 

\begin{abstract}
While many relate stock market trading to gambling, successful traders have turned stock picking into a science. The likes of Warren Buffet tell us that successful stock buying is all in the research. So what kind of research aids in the prediction of companies within the highly volatile pharmaceutical market? The use of available, open-source APIs and Google Alerts are used to explore if clinical trial results can directly impact stock performance in small, mid, and large cap pharmaceutical companies. Key words and/or phrases in results and related news articles are identified as possible predictors of market effect. As well as a comparison to already established analyst ratings from Barclays, Goldman, Morningstar, or others which have already been shown to impact stock performance.
\end{abstract}

\keywords{Big Data, HID313, i523, Stock Market, Pharmaceutical}
\maketitle
\section{Introduction}
A ``stock'' is a piece of ownership in a company. Offering stocks for sale provides capital to the selling company in exchange for a stake in the company. A stock market is a collection of exchanges where trading of stocks takes place \cite{www-investopedia}. Evidence of early stock markets date back to the fourteenth century with the offering of state loan stocks throughout Italy. Even prior to the organization of stock markets, price fluctuations for goods such as wheat and barley were tracked by early economists. The first ``modern'' stock market appeared in Amsterdam in the seventeenth century where the volume of stocks traded and the fluidity in which they were traded reached a new high \cite{Braudel}. 

The biggest stock markets in the world are currently the New York Stock Exchange (NYSE), the National Association of Securities Dealers Automated Quotations (NASDAQ), and the London Stock Exchange. NYSE did stuff... NASDAQ began as an all-electric equities exchange in 1971 and today provides trading, technology, and information services for financial markets. Today   \cite{www-nasdaq}.

Throughout the history of markets, prices have been tracked and insightful traders have attempted to predict and capitalize on price fluctuation. The age of computers opened new doors for stock analysis and trend prediction to facilitate capital gains for traders. Financial companies like Goldman Sachs and JPMorgan Chase \& Co. have hired mathematicians, statisticians, and trade analysts since the early days of trading in an effort to predict the market in a consistent manner. %need cite for financial comapnies use to algorithms - look up history
Once an algorithm is established and used consistently the algorithm itself but be considered as a variable that could effect the prediction outcome \cite{Hellstrom}. 

A major complexity in creating algorithms for the stock market is that the market tends to follow the erratic emotions and feelings of humans. If computers were running the market, making trade decisions based on logic and reason, then the market would be much more stable. The volatility of human emotions about money and stocks creates tremendous volatility in the market. The revolution of social media has provided a means of measuring the mood of possible traders \cite{BOLLEN}.

%Explain API, Machine Learning Tech,Text Mining  how they are used what they are
An Application Programming Interface (API) acts as the middleman between the requesting service and the preforming service. When a user or system submits a request the request is passed to the API which translates it for the processing system then returns the results in a receivable format. 


\subsection{Pharmaceutical Sector}
The pharmaceutical industry %is driven by stuff and things

Like the financial sector trying to predict the stock market, the pharmaceutical industry has devoted resources to developing prediction algorithms and machine learning systems. The efforts of drug manufacturers are to create a system that consistently predicts or aids in identifying drug targets. %talk about virtual high throughput



\subsection{Clinical Trials}
A clinical trial is a planned experiment involving patients with the intent to elucidate an appropriate or effective treatment option(s) for the population of patients afflicted with the same medical condition. A big concern with clinical trials is that inferences are made for the entire population of patients from a relatively small sample size \cite{Pocock}. One of the first clinical trials recorded was carried out in the eighteenth century to evaluate six treatments on  twelve patients with scurvy. Two patients that were given oranges and lemons recovered very quickly. Fisher introduced the concept of randomization in the nineteenth century \cite{Friedman}.

Clinical trials have four defined phases. Phase I trials identify how well a drug is tolerated by determining the maximally tolerated doe (MTD) on a very small sample size. Phase I trials have very simple experimental designs as the only intent is to examine toxicity. Phase II explores biological activity or effect on a small patient sample size. The design of a Phase II trial is dependent on the design on the Phase I trial as both share the intent to evaluate adverse events. Phase III trials follow the design of Phase II trials but on a bigger sample size with the intent to solidify a treatment's effectiveness in clinical practice \cite{Friedman}. 

Clinical trial designs have been very slow to evolve due to %stuff and things restricting clinical trials

\subsection{Established Analyst Ratings}
%Where did these rankings come from?

%Goldman

%Chase

%Morningstar



\subsection{Data Resources}
NASDAQ's website provides historical stock performance data that can be exported as a Comma-Separated Values (CSV) file. The disadvantage of NASDAQ's free export service is that each stock must be exported separately. The free quote service can be accessed at http://www.nasdaq.com/quotes/. NASDAQ provides API services for subscribers starting at \$5,000 per year \cite{www-nasdaq-sub}. Access to NASDAQ's API services can also be granted through corporate sponsorship. NASDAQ's free CSV export services were used to collect initial project data. In example, the stock history for Celsion Corporation during the week of August 21, 2017 is shown.

\begin{mdframed}[style=default]
\begin{lstlisting}
date,close,volume,open,high,low
2017/08/25,1.3700,179097.0000,1.3600,1.4100,1.3000
2017/08/24,1.3600,149832.0000,1.3100,1.3600,1.2810
2017/08/23,1.3100,223451.0000,1.2500,1.3300,1.2430
2017/08/22,1.2800,164594.0000,1.3200,1.3200,1.2400
2017/08/21,1.3300,169037.0000,1.3300,1.3700,1.2800
\end{lstlisting}
\end{mdframed} 

Exports such as this one offered by NASDAQ and API interfaces for stock data are provided by numerous companies. The Yahoo! Finance API is explored below and the Google Finance API was used to perform the stock data extraction for the analysis presented. Additional resources such as stock tracking apps and free exports are available. CSV exports such as the one listed above can be downloaded from Google Finance, Yahoo! Finance, and many others. This publication does not provide a complete list of available resources, but attempts to present a few for comparison. 

Python.org provides a python module to pull stock data from Yahoo! Finance \cite{www-python-yahoo}. The package can be installed through Git by cloning the Git directory where the package is available: https://github.com/lukaszbanasiak/yahoo-finance.git. To install the python package without Git the tape archive can be downloaded from https://pypi.python.org/pypi/yahoo-finance. While Yahoo! Finance is a great resource, the API does not function consistently. To install the python package without Git, the tape archive can be downloaded from https://pypi.python.org/pypi/yahoo-finance. Tape archives allow for compression of multiple files which can be restored to their original format using the tar command in the command line \cite{www-tar}.  Apply the tar options: z - filter archive through gzip, x - extract an archive file, and f - filename of archive, use ``cd'' to change the current working directory, and then install the python module using the package management command ``pip'':
\begin{mdframed}[style=default]
\begin{lstlisting}
tar -zxf yahoo-finance-1.4.0.tar.gz
cd yahoo-finance
pip install yahoo-finance
\end{lstlisting}
\end{mdframed}


\section{Methods}
\subsection{Data Collection}
%news articles from Google Alerts for text mining
Data collection was initiated with the use to Google Alerts. Google allows for alerts to be configures from https://www.google.com/alerts. Gmail users can configure these alerts to be sent through email when news or other types of articles are released to the web. The Google Alerts for this project were: ``Phase III Trial'', ``Phase 3 Trial'', and "``Meets Primary End Point''. When these phrases are detected by google, the link to the webpage and a short description are sent via email to the configured email address. On busy days, an excess of 100 alerts were received for this data. On slow days, only a couple alerts were received.  Only very infrequently were no messages received.

To collect data from the received Google Alerts without too much manual clicking, Gmail has an available API which allows users to pull data from a Gmail account. To start using the Gmail API a user must first configure their Authentication credentials through Google's developer console. Once credentials are received the form of a JSON file, the Google Client Library can be installed using pip to install google-api-python-client. The Google Development team has provided a quickstart file which facilitates the first authentication run. Running this quick start guide will open a browser window and prompt the user to log into a Gmail account. The user then accepts the authorization and can run the Gmail API from command line or other compilers. 

 Headlines of the received alerts, usually the title of the article and the first couple of lines, are referred to as ``Snippets'' by Google's Gmail API. This project pulled only the Snippets and the date from the Google Alerts. The Snippets do not contain the whole article but may still provide enough evidence of sentiment for further analysis and prediction of the associated stock. Unfortunately, no solution was identified for extracting the appropriate stock symbols from the Snippets so this task had to be performed manually. 

%daily stock performance

The Python package pandas is an incredible resource 
%explain Pandas
The Pandas package has a resource for getting stock market data from free online sources such as Yahoo! mentioned above and Google. To install this package through Git, simply clone the directory, use the ``Change Directory'' command ``cd'' to change the current working directory, and installing the python module as follows: 

\begin{mdframed}[style=default]
\begin{lstlisting}
$ git clone git://github.com/pydata/pandas-datareader.git
$ cd pandas-datareader
$ python setup.py install
\end{lstlisting}
\end{mdframed}

If the Python setup returns the error: ``python: command not found'' run the following with the path to the python installation:
\begin{mdframed}[style=default]
\begin{lstlisting}
$ PATH="$PATH:/c/Python27"
\end{lstlisting}
\end{mdframed}

Pandas-datareader and many other packages can also be installed via pip. In example, many additional packages are needed to run a python script using pandas-datareader. These packages can be configured all at once or one at a time as follows:
\begin{mdframed}[style=default]
\begin{lstlisting}
pip -m install --user numpy scipy matplotlib ipython jupyter pandas sympy nose urllib3 chardet idna
\end{lstlisting}
\end{mdframed}

Unlike the NASDAQ export, using Google as a data source for pandas-datareader requires each attribute to be called separately. This means calling the Close Price, Open Price, High Price, etc individually and joining them through code. Also, unlike NASDAQ's export but this time in a positive light, multiple tickers can be passed together. This allows for all historical data to be pulled for many stocks with a single code. 

\begin{figure}[htb]
\begin{verbatim}
your code here 
\end{verbatim}
\caption{Your caption}\label{c:codexyz}
\end{figure}

Figure \ref{c:codexyz} shows the code to download and extract the data.


%trial results


\subsection{Data Analysis}
%text mining

%logistic regression


\section{Results}

\subsection{Comparison to Established Analyst Ratings}
%What did the analysts perdict about the companies analyzed?


\section{Conclusion}


\begin{acks}

The author would like to thank Dr. Gregor von Laszewski and the teaching assistants of the Fall 2017 i523 course for their support and suggestions to write this paper.

\end{acks}

\bibliographystyle{ACM-Reference-Format}
\bibliography{report} 


\end{document}
